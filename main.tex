\documentclass[pastel]{simplehipstercv}
\usepackage[utf8]{inputenc}
\usepackage[default]{raleway}
\usepackage[margin=1cm, a4paper]{geometry}
\usepackage{hyperref} % Ajout du package hyperref

\newlength{\rightcolwidth}
\newlength{\leftcolwidth}
\setlength{\leftcolwidth}{0.23\textwidth}
\setlength{\rightcolwidth}{0.75\textwidth}

\title{Axel Jacquin's CV}
\author{\LaTeX{} Axel Jacquin}
\date{\today}

\pagestyle{empty}
\begin{document}

\thispagestyle{empty}

\section*{Début}

\simpleheader{headercolour}{Axel}{Jacquin}{Ingénieur en Électronique Numérique}{white}

\subsection*{}
\vspace{4em}

\setlength{\columnsep}{1.5cm}
\columnratio{0.23}[0.75]
\begin{paracol}{2}
\hbadness5000

\paracolbackgroundoptions

\footnotesize
{\setasidefontcolour
\flushright

\begin{center}
    \roundpic{pictures/axel.jpg}
\end{center}

\bg{cvgreen}{white}{Coordonnées} \\[0.5em]

\icon{\faEnvelopeO}{cvgreen}{} 4 bis, Av. Antoine Ravel

\icon{\faMapMarker}{cvgreen}{} 42330 Saint-Galmier

\icon{\faPhone}{cvgreen}{} 06.40.35.01.25 

\protect\url{axel.c.e.jacquin@gmail.com}

\bigskip

\bg{cvgreen}{white}{Compétences}\\[0.5em]

\texttt{Verilog} ~/~ \texttt{SystemVerilogV} ~/~ \texttt{VHDL}

\texttt{Xsim} ~/~ \texttt{Modelsim} ~/~ \texttt{Gtkwave}

\texttt{Vivado} ~/~ \texttt{VSCode} ~/~ \texttt{Netbeans}

\texttt{Git} ~/~ \texttt{Mercurial}

\texttt{Windows} ~/~ \texttt{Linux}

\texttt{Traitement d'image} ~/~ \texttt{IA}

\bigskip

\bg{cvgreen}{white}{Centres d'intérêt}\\[0.5em]

Robotique, Rubik’s Cube

MAO, Piano

Tennis, Course à pied, Escalade, Judo

\vspace{41em}

\infobubble{\faGithub}{cvgreen}{white}{\href{https://github.com/AJacquin}{AJacquin}}

\infobubble{\faLinkedin}{cvgreen}{white}{\href{https://www.linkedin.com/in/axeljacquin}{axeljacquin}}

\phantom{tournez la page}

}
\switchcolumn

\small
\section*{Expériences}

\begin{tabular}{r| p{0.5\textwidth} c}
    \cvevent{2021--}{Ingénieur en Électronique Numérique}{\href{https://aem.com.sg}{AEM Mu-Test}}{Saint-Jean-Bonnefonds (France)}{Architecture, développement, simulation et validation FPGA.

    Architecture et développement de logiciels débarqués.
    
    Mise en place d'un serveur de tests de non-regressions.

    Caractérisation d'un système de test de composants électroniques.

    Planification du développement d'une carte électronique.
    
    FPGA Xilinx, PICmicro, Verilog, UVM, Python, cocotb, C, C++, Java, Hg, Planification}{pictures/aem.png} \\
    \cvevent{2020--2021}{Ingénieur en Automatisme}{\href{https://www.knds.fr/en/our-brand/knds-cbrn}{NBC-Sys (KNDS CBRN)}}{Saint-Chamond (France)}{Programmation d’une IHM et de schémas électriques pour des véhicules blindés.
    
    Bus CAN, Javascript, Documentation}{pictures/nbc.png} \\
    \cvevent{2020}{Ingénieur en Systèmes Embarqués}{\href{https://www.ciose.fr/}{CIO Systèmes Embarqués}}{Saint-Etienne (France)}{Conception d’un système d’étalonnage et détection des pertes de connectivité pour un pénétromètre.
    
    STM32, C, Python, wxWidgets, Git, Électronique analogique}{pictures/cio.png} \\
    \cvevent{2019}{Ingénieur en Systèmes Embarqués}{\href{https://www.tudublin.ie}{Dublin Institute of Technology}}{Dublin (Ireland)}{Conception d’un scanner Bluetooth (BLE), vulgarisation de la domotique à des personnes âgées.
    
    Raspberry Pi, Python, Linux, Anglais}{pictures/dit.jpg} \\
    \cvevent{2017}{Technicien en Automatisme}{\href{https://www.o-i.com}{O-I Manufacturing}}{Veauche (France)}{Conception d’un banc d’essai pour des machines de formage de bouteilles en verre.
    
    Automates (Télémécanique), Ladder, Electrotechnique, Mecanique, Maintenance}{pictures/oi.png}
\end{tabular}
\vspace{3em}

\begin{minipage}[t]{0.35\textwidth}
\section*{Diplômes}
\begin{tabular}{r p{0.6\textwidth} c}
    \cvdegree{2020}{Diplome d’Ingénieur}{Imagerie et systèmes électroniques}{Télécom Saint-Etienne}{France}{pictures/tse.png} \\
    \cvdegree{2020}{Master EEEA}{Electronique, Energie Electrique et Automatique}{Ecole Centrale de Lyon}{France}{pictures/ecl.png} \\
    \cvdegree{2017}{DUT GEII}{Génie Electrique et Informatique Industrielle}{IUT de Saint-Etienne}{France}{pictures/iutse.png} \\
    \cvdegree{2017}{DU CITISE}{Cycle Initial en Technologies de l’Information}{Télécom Saint-Etienne}{France}{pictures/tse.png}
\end{tabular}
\end{minipage}\hfill
\begin{minipage}[t]{0.3\textwidth}
\section*{Programmation}
\begin{tabular}{r @{\hspace{0.5em}}l}
    \textbf{Verilog} &  \barrule{0.55}{0.5em}{cvgreen}\\
    \textbf{VHDL} & \barrule{0.45}{0.5em}{cvgreen} \\
    \textbf{C, C++} & \barrule{0.5}{0.5em}{cvgreen} \\
    \textbf{Java} & \barrule{0.2}{0.5em}{cvpurple} \\
    \textbf{Javascript} & \barrule{0.25}{0.5em}{cvpurple} \\
    \textbf{Python} & \barrule{0.45}{0.5em}{cvgreen} \\
    \textbf{Matlab} & \barrule{0.3}{0.5em}{cvpurple} \\
    \textbf{\href{https://github.com/AJacquin/CV}{\LaTeX}} & \barrule{0.25}{0.5em}{cvpurple} \\
\end{tabular}

\vspace{6em}

\section*{Langues}
\begin{tabular}{l | ll}
\textbf{Français} & C2 & \pictofraction{\faCircle}{cvgreen}{4}{cvgreen}{1}{\tiny} \\
\textbf{Anglais} & C1 & \pictofraction{\faCircle}{cvgreen}{4}{black!30}{1}{\tiny} \\
\textbf{Espagnol} & B1 & \pictofraction{\faCircle}{cvgreen}{3}{black!30}{2}{\tiny} \\
\textbf{Portugais} & A1 & \pictofraction{\faCircle}{cvgreen}{1}{black!30}{4}{\tiny}
\end{tabular}

\end{minipage}

\end{paracol}

\end{document}
